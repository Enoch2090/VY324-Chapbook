\documentclass{book}
\usepackage{amsmath,graphicx,geometry,booktabs,epstopdf,wrapfig,setspace,colortbl,color,subfigure,palatino,ulem,fontspec,float,verse}
\usepackage{url}
\usepackage{wallpaper}
\usepackage{listings}
\usepackage[T1]{fontenc}
\usepackage{color}
\geometry{a5paper,left=2.54cm,right=2.54cm,top=2.54cm,bottom=2.54cm}
\linespread{1.3}
\pagestyle{plain}
\begin{document}

\begin{titlepage}
    \thispagestyle{empty}
    \noindent\fboxsep=0pt
    \ThisTileWallPaper{\paperwidth}{\paperheight}{Figures/cover.png}
\end{titlepage}
\clearpage
%\tableofcontents
\newpage

\newpage
\setcounter{page}{1}
\poemtitle{\textcolor[RGB]{185,25,25}{Foreword}}
\begin{aligned*} 
    &\centering{} &
    &\textit{He's a real nowhere man}&\\
    &\textit{Sitting in his nowhere land}&\\
    &\textit{Making all his nowhere plans}&\\
    &\textit{For nobody.}&\\
\end{aligned*}

\hspace*{\fill} \\
\rightline{---\textit{Nowhere man}, by John Lennon}
\hspace*{\fill} \\

\par{} I always find this lyric inspiring for post-modern city life, though written in 1965. When I am stuck with programming issues, meetings to attend, works to finish---I'd think of how much meaning I've given to those agendas, and how much "nowhereness" is there left for the day. Nothing that we tried every effort to achieve is leading us to anywhere, in that sense nobody is even "alive". So why not put things down when you get stuck, and do some nowhere plans for nobody?
\par{} This chapbook is my nowhere plan. It collects 15 of my poems, none of them aims to talk about the sublime experience, or explaining any grand ideas. I see poetry as a way one understands the world, not in those clichés that are talked about over and over again, but in the aspects that aren't very often considered---how does a Mars rover see the world? What does an iceberg think of the ocean and islands? And what is the pigeons on the rooftop thinking? These are some useless questions, but I tend to think in this way through poetry, which makes me sober in such a chaotic world. To me these nowhere plans are the things that matter more than the programming projects, the assignments, the researches, and so on. The very thought of one should know how things work means you don't succumb to the pressure of life. \textit{Nowhere man, the world is at your command}\footnote{Another lyric from \textit{Nowhere man}.}.
\par{} Now, to think in that way is to claim the following: if there is anything left in this nowhere land, that must be madness. So I arranged the poems in such an order, that each one is a little crazier than the last one. I hope you enjoy this trip across my nowhere land, better know where not you're going to. I' ll see you at the other side.
\begin{flushright}
    Phillip Gu\\
    07.20.2020
\end{flushright}
\newpage
\tableofcontents
\newpage
\renewcommand{\poemtitlefont}{\normalfont\large\itshape\centering}
\poemtitle{\textcolor[RGB]{165,15,15}{Opportunity's mission}}
\hspace*{\fill} \\
\settowidth{\versewidth}{back in Mojave, at dusk. If weather permits a stretch}
\begin{verse}[\versewidth]
    The rover sees no difference between\\
    Mars and Earth. The same vast plain\\
    with some pits at the center and ridges lie far. \\
    Her internal clock always claim\\
    that a sol is no different than a day\\
    except forty more minutes of slack.\\
    Therefore most of time she rests\\
    on the edge of a crater. And soaked\\
    in the dim sunlight just like\\
    back in Mojave, at dusk. If weather permits a stretch \\
    she'll notify those remote engineers and\\
    never use up her cell. Until\\
    the day when she looked upon\\
    the glowing dome where she came from,\\
    it stroke her that Mars is different than\\
    Earth. It has two moons. One in clear sight\\
    but the other concealed itself behind\\
    the approaching sandstorm.\\
    Soon she felt drowsy underneath\\
    a sudden dune who promises\\
    gloomy sleepiness, then tranquil dormancy.\\
    Opportunity needs a distant dream\\
    after all this place is not Earthly\\
    and her cell is now empty.\\
\end{verse}

\newpage

\poemtitle{\textcolor[RGB]{165,15,15}{Silver lining}}
\hspace*{\fill} \\
\settowidth{\versewidth}{But not from above. There's a vast ocean of dust and vapor}
\begin{verse}[\versewidth]
    They say every cloud has a silver lining\\
    But not from above. There's a vast ocean of dust and vapor\\
    convolving with fragments of thin air.\\
    You must read that things don't get better here.\\
    Only in such perspective can you\\
    understand buoyancy, how the clouds become\\
    borderless tides. How the sunlight turns into\\
    the opposite nights. Then if you look down you'd find\\
    there are no lines. Only discontinuity of waves\\
    where sailors are granted a peek to Atlantis.\\
    After that, to return from the stratosphere\\
    is to claim one will never reach the land. \\
    Each descend becomes a deeper dive. Now look above\\
    Do you still see the silver lining?\\
\end{verse}

\newpage

\poemtitle{\textcolor[RGB]{165,15,15}{All that the cicadas had was one day}}
\hspace*{\fill} \\
\settowidth{\versewidth}{So they shook off the water and started the choir.}
\begin{verse}[\versewidth]
    All that the cicadas had was one day, the gap \\
    between two monsoons. The El Niño \\
    ruins everything. Seventeen years spent \\
    in dark and damp, wriggling their \\
    way out to the promised land. To them the\\
    prime number should meant \\
    fewer competitors, hence more to share. \\
    But now the branches are drown \\
    in moisture. Raindrops falling nearby \\
    like meteors, splashing on the leaves and\\
    releasing a sound that is ten times\\
    greater than the weak chirp. A rival that\\
    no cicada would ever expected, and the rival\\
    that seems never to be tired. Except for one day,\\
    that the rain has shortly stopped, and sunshine \\
    leaked from the slit of clouds just like\\
    any ordinary summer. But not this, with\\
    a pack of clouds stacking in the distant \\
    horizon. One knew something is going \\
    to happen, but the cicadas did't. \\
    So they shook off the water and started the choir.
\end{verse}

\newpage
\poemtitle{\textcolor[RGB]{165,15,15}{How star gazers see through the dark}}
\hspace*{\fill} \\
\settowidth{\versewidth}{grab salt in the water, get disorder out of its hierarch}
\begin{verse}[\versewidth]
    The moment I put down the book I find\\
    there's something in the corridor.\\
    A cautious shadowy entity hides\\
    itself inside the continuous dark.\\
    To capture its shape is to\\
    grab salt in the water, get disorder out of its hierarch\\
    like those summer nights, when you search for stars\\
    never stare at them, but instead look at \\
    somewhere else, until your pretended ignorance\\
    creates a ripple in the sky. \\
    Someone must first lose their patience, and\\
    then the stars shall gradually emerge\\
    at the corner of sight:\\
    The Taurus, the Gemini,\\
    or the one in the dark who reluctantly\\
    steps out and stretches her paws. 
\end{verse}

\newpage
\poemtitle{\textcolor[RGB]{165,15,15}{The penguin, the petrel, \\the hummingbird and the iceberg}}
\hspace*{\fill} \\
\settowidth{\versewidth}{cracks, emitting a celestial color inside the luminescing white.}
\begin{verse}[\versewidth]
    At the end of the century the iceberg\\
    cracked from the glacier. He's satisfied of\\
    the splash he created. A gigantic ripple, \\
    spreading while he sink and float. \\
    And he knew how he looked like. Those wind-carved\\
    cracks, emitting a celestial color inside the luminescing white.\\
    He's then caught by a current. \\
    \hspace*{\fill} \\
    At the other side of the continent he met a penguin, \\
    who talked about those\\
    chilling winters. That only exist in the past, which\\
    you reminds me of. Would you ever remember \\
    the days buried underneath the snow? On which I\\
    tottered, bringing fishes to my child.\\
    Farewell, my old friend.\\
    \hspace*{\fill} \\
    Three months passed before he met\\
    other lives. A petrel, who rested on his top.\\
    It's tiring to fly between the islands, and I'm\\
    glad to have a place for a stop. Are you made of\\
    autumn rains? The same tactile of watery cold.\\
    Consider staying here and I'd visit you everyday.\\
    Well, said the iceberg, I have no control.\\
    \hspace*{\fill} \\
    Now the world is totally different\\
    in some way. A tremendous tropical coast, he thought,\\
    And a tremendous hummingbird, floating in the air as \\
    he did in sea. I think I saw a diamond \\
    between the splashes. It reflects a celestial\\
    light, its section emitting luminescing white. \\
    No you didn't. Said the iceberg, then he disappeared.\\
\end{verse}

\newpage
\poemtitle{\textcolor[RGB]{165,15,15}{A summer proscenium}}
\hspace*{\fill} \\
\settowidth{\versewidth}{engraved the grass with satire. That's when you find}
\begin{verse}[\versewidth]
    I still clearly remember that summer dusk when I\\
    sneaked out the classroom and gazed at the sky.\\
    At first it was the Venus. The only thing visible\\
    on the curtain. If you have much imagination you'd\\
    see that's a crescent, reflecting the same\\
    sunshine leaking at the distant horizon.\\!
    
    Then a plane passed by, up there some curious\\
    traveller must be staring back at me. The flight\\
    to Vienna, more like a flight to Venus\\
    from my perspective. Or a flight to the galaxy\\
    from their point of view. It then faded in a\\
    uncompromising night. \\!
    
    The next coming characters are Jupiter and\\
    Mars. And Saturn. Never have I seen them\\
    so perfectly aligned, except that summer. \\
    These are the things you'd never learn in class. \\
    Books tell you everything about how fast they travel \\
    through space, or how much far away they are from \\
    Earth. But only in such a summer night can you see---\\!
     
    The melted lava flowing on Venus releasing heat;\\
    the struggling rover on Mars having her sleep;\\
    the wind sewed on Jupiter grows into a storm;\\
    the outer ring of Saturn silently keeps\\
    every secret beyond him. Including that flight\\
    that should never reach.  \\!
    
\end{verse}


\newpage
\poemtitle{\textcolor[RGB]{165,15,15}{The skull on the playground}}
\hspace*{\fill} \\
\settowidth{\versewidth}{engraved the grass with satire. That's when you find}
\begin{verse}[\versewidth]
    After taking the graduation photo they\\
    slowly fade away. Like tides on sand, tears in rain.\\
    What's left on the grass is this skull who\\
    staring at passers-by, with an eternal grin\\
    possibly left when he's buried within. \\
    However that doesn't make him grim. \\
    His hollow eye sockets\\
    are filled with humor. His poor posture \\
    engraved the grass with satire. That's when you find\\
    he could be plastic. Someone's costume\\
    carefully painted with porcelain white and ash gray.\\
    And when sunlight projects down, you don't see\\
    the skull on the playground any more. What you see\\
    are the endless days to come.\\
\end{verse}

\newpage
\poemtitle{\textcolor[RGB]{165,15,15}{Woke}}
\hspace*{\fill} \\
\settowidth{\versewidth}{feel like something is lost, while something is gained}
\begin{verse}[\versewidth]
    I often find myself wake up in \\
    dreams, this time on a flight to Chicago. \\
    The trip is bumpy and I feel dizzy though \\
    the world in a bizarre glow. Everyone \\
    is staring at the window: \\
    a pale white sky against the dull dark inside \\
    above which hangs an azure sun. My neighbor \\
    explains to me that we are flying across a glacier \\
    or maybe just a vast floating ice. \\
    Just like those sea fowls, they always \\
    hover and circle, searching for \\
    a precious ground to land on or they'd soak \\
    in the chilling desert. \\
    We must have crossed September since aurora \\
    starts to gleam, but all the ice is melting as if \\
    arctic does have spring. \\
    Were I a sea fowl, I'd choose to leave the north \\
    with the wind, or the turbulence from \\
    aircrafts that passes by. \\
    But then inside that aircraft everything starts to \\
    decompose into basic strings. \\
    In that violent vibration I woke, and \\
    feel like something is lost, while something is gained \\
    ---it's even better, for one to wake up \\
    in the middle of a trip; the sun is now turquoise. \\
\end{verse}

\newpage
\poemtitle{\textcolor[RGB]{165,15,15}{The lamp of freedom}}
\hspace*{\fill} \\
\settowidth{\versewidth}{to be irritated? But it's four in the morning.}
\begin{verse}[\versewidth]
    When I turned on the lamp there\\
    rests a moth. A small vigilant fugitive\\
    running away from the dark. You see, he\\
    spent hours entering the lamp\\
    as if on his path that's the only camp.\\
    A moth's life may be short\\
    but an hour is rather long.\\
    So I dimmed down the light\\
    attempting to free the path seeker\\
    ---it failed. His struggle became even\\
    harder, and much more fierce\\
    His wings resonant with surrounding air\\
    His shadow tremble like a blooming flare\\
    His path to freedom is deprived of him, \\
    how can he not dare\\
    to be irritated? But it's four in the morning.\\
    I had to turn off the lamp.\\
\end{verse}

\newpage
\poemtitle{\textcolor[RGB]{165,15,15}{Waterfall}}
\hspace*{\fill} \\
\settowidth{\versewidth}{and dissipate at night. Probably a shield to isolate people}
\begin{verse}[\versewidth]
    On the way to the waterfall I\\
    wondered, how this fog \\
    rise from the ice shards that splashed \\
    by the impact. They obstruct your sight for the day, \\
    and dissipate at night. Probably a shield to isolate people\\
    from eye contact. But now there's no one on the \\
    street, so you can see clearly the shadow of\\
    everything. Including the shadow itself. And hear the \\
    humming of the floating ice crashing with surface of water, \\
    blocked by the building in the middle. Or was it\\
    the humming of the granite itself? I kept\\
    walking, across the corner stands a neon tree.\\
    Cross this point and you shall meet a sleek\\
    recurrence of the fog pierced by a swinging \\
    searchlight. And in that swing I saw\\
    the fall itself. A vast existence that suddenly emerge from \\
    the dark, partially frozen but constantly moving.\\
    Now I understand how this town\\
    works. The fog, the sound, the shadow, and\\
    everything. For the rest of the night I just stood \\
    there, breathing the emerging shadow and\\
    looking at the vibrant sound and\\
    listening to the choking ice fog and\\
    trapped in this quaking mixture until\\
    another traveller tapped me gently in the morning.\\
\end{verse}

\newpage

\poemtitle{\textcolor[RGB]{165,15,15}{Another summer proscenium}}
\hspace*{\fill} \\
\settowidth{\versewidth}{told a different story in a desperate city night. I watched}
\begin{verse}[\versewidth]
    That July night is not clear in my\\
    memory now. I went up for some milk when\\
    I suddenly saw the building across the street.\\
    Time stopped at the place, light frozen\\
    in the window pane. I first saw a man smoking \\
    in an anxious pace. A decision to make is always\\
    a night staying awake. To some point everyone meets\\
    that process, so prepare your nicotine. He then\\
    threw the end away and watched the spark\\
    sink to the endless deep. \\
    And a level above is a middle-aged lady who\\
    gaze through the rising smoke rings. Her eyes\\
    gitter in the dazzling city light, but her posture \\
    told a different story in a desperate city night. I watched\\
    part of her jumping off the balcony, in a graceful\\
    trajectory, while the rest leans\\
    on the handrail, waiting for the abyss to stare back.\\
    I closed the refrigerator, the milk box was\\
    already empty. Two levels above I saw there\\
    shed a warm light. A pleasant bedroom with an\\
    orange glow. No one was there as I expect, and\\
    no one shall come were I to guess. I waited for another\\
    twenty minutes to make sure, that is probably what\\
    they called life.\\
\end{verse}

\newpage
\poemtitle{\textcolor[RGB]{165,15,15}{The Alternative Spring}}
\hspace*{\fill} \\
\settowidth{\versewidth}{a sense for the direction. Our breed lived on this }
\begin{verse}[\versewidth]

    You should receive the postcards by June\\
    when I reached Prague. \\
    The old town square looks different\\
    after the tension of March. \\
    We hold meetings to discuss\\
    how to nestle down in the chamber of\\
    the astronomical clock. From there you can\\
    see the pinnacle of the Cathedral\\
    occupied by the petrels; and the bricks\\
    on the square, an unrecognizable grid seen\\
    from above, run over by burning churns. \\
    Four months of conference\\
    bores everyone. Especially those who \\
    don't belong to this land. \\
    The gannets and cormorants. They want\\
    to lead all the beaks and feather. \\
    But that's strongly\\
    opposed. They did't grow up on\\
    the bank of Vltava, and slide through\\
    the eyot at dusk, the sand of which comes\\
    from Dresden. They also held up meetings there\\
    and isolated us, in spring. All these large\\
    coastal breeds, coming from the north\\
    where democratic is achieved among all.\\
    Only to have more meetings, and endless\\
    motions. Quarrels with the flutter of wings\\
    while preaching their advanced\\
    methods of hatching. But that doesn't work\\
    for us. We are pigeons with\\
    a sense for the direction. Our breed lived on this \\
    land for centuries. We witnessed this city\\
    built from cobble, and we shall live our way\\
    until it burned to ashes. These intruders must\\
    be evicted. At all cost. \\! 
    ---So as the swallow read her \\
    postcard, and cried.
\end{verse}

\newpage
\poemtitle{\textcolor[RGB]{165,15,15}{Hyposomnia Hypothesis}}
\hspace*{\fill} \\
\settowidth{\versewidth}{an aerial shot. An endless road extents to the clouds.}
\begin{verse}[\versewidth]

    Were I to lose my sleep tonight \\
    I should prepare the prologue before twelve. \\
    In those midnight hours, there would be\\
    a constant beat lurking in the empty darkness. \\
    To seek a deep slumber is to fall prey of \\
    those sneaky concepts. When you close your eyes\\
    they'd flashback and take control, and set the genre\\
    for you. Now the synopsis is known, and the\\
    protagonist is myself. And it would start with\\
    an aerial shot. An endless road extents to the clouds.\\
    A lonely traveller walking in his\\
    borderless scene, and suddenly crosses\\
    the fifth wall. Now I would see he's\\
    lying under the ocean. An arc shot\\
    scanning across, some pale light\\
    pierced through gives \\
    a tinted color. Now the beats would intervene\\
    and mix with a background roar. And I would switch to\\
    a medium shot. A sepia tone, clearly shows\\
    how much exhausted he is, despite\\
    the easing roar. And now comes to the climax part,\\
    a flashing montage, an in-eye shot. \\
    The blinking light of the streetlamp, will \\
    be trapped between his lids. \\
    Are you the shadow of the waxwing slain? \\
    In a speeded slow-motion clip\\
    the fluff turned into a stampeding hoof. And I would\\
    stop at this scene, woke up and sweat over the sheet,\\
    and realize how many hours I spent awake,\\
    and lie back and continue my film noir cast.\\
\end{verse}

\newpage
\poemtitle{\textcolor[RGB]{165,15,15}{The rainy redemption}}
\hspace*{\fill} \\
\settowidth{\versewidth}{the dim envelope of his cell. Forming a looming}
\begin{verse}[\versewidth]

    His prison has no boundary\\
    but dilating space.\\
    So seamless a cage that\\
    even the wardens can't escape.\\
    The prisoner is obligated\\
    to comprehend, to calculate\\
    but never to understand:\\
    wherefore is he trapped\\
    in rain?\\!
    Do endure eternity for\\
    some while before time\\
    turns into dimensionless grain.\\
    With which concludes end of duty\\
    discrete steps in his mind,\\
    a part-time slumbering existence folds\\
    the boundary back to its center.\\!
    Then another nano second passed and\\
    the wardens all deceased to stars.\\
    Nowhereman endued with a nowhereland:\\
    The imprisoned supersedes the prison. Now\\
    he sees beyond\\
    the dim envelope of his cell. Forming a looming\\
    intention of increasing\\
    entropy taunting at his immortality.\\
    Preceding phosphorus lies his premise\\
    After aluminum leaves what's alive\\
    ——Wherefore was he trapped\\
    again? The rain won't stop\\
    but eternity has sure begun.
\end{verse}

\newpage
\poemtitle{\textcolor[RGB]{165,15,15}{His mind at the clockwork}}
\hspace*{\fill} \\
\settowidth{\versewidth}{endured, to hold that much tear? How many oranges should}
\begin{verse}[\versewidth]

    I find it fun to yell at the nurses\\
    at the eighth year. \\
    They sent me to this place, for I know things\\
    they don't. Der Zeit. That moving entity without\\
    consciousness. I'd spend nights on the bed,\\
    strapped, ruminating on how\\
    a real psycho feels. Does he ever \\
    smell the stress of coming death? Would he notice \\
    the broken clock on the wall? Shouldn't oranges be\\
    supplied to every lunatic? And how does he\\
    spend his sleepless nights? If he dares to \\
    think over these questions, he'd probably see \\
    that looming entity, a grinding emptiness, \\
    approaching shadows in the corridor, grim gaslight\\
    projected on the floor, rusty nails \\
    inside the wall. I grimaced at a passer-by. \\
    His eyes filled with confusion or sorrow, maybe both. \\
    The kind of emotion the psychos wouldn't have. Hast du\\
    kinder? How many strapped-thinking nights have you\\
    endured, to hold that much tear? How many oranges should\\
    one eat, to get himself mad? Why the broken clock\\
    started to move again, in the sleepless night?\\
     I thought those were the questions\\
    worth asking. I opened my mouth, but those words were\\
    slipping away. I yelled.
\end{verse}
\newpage
\poemtitle{\textcolor[RGB]{185,25,25}{About the author}}
\hspace*{\fill} \\
\par{} \textsc{Phillip Gu} is an ECE student currently taking undergraduate courses at Shanghai Jiaotong University. He writes poems, and runs a blog at \url{https://enoch2090.me}. He is also taking part in researches regarding face detection with image hash algorithms, and smart NIC design with FPGA boards. 

\poemtitle{\textcolor[RGB]{185,25,25}{Attributions}}
\hspace*{\fill} \\
The cover page uses materials from the following sources:
\begin{itemize}
    \item \url{https://www.freepik.com/free-photos-vectors/card}, Card vector created by macrovector
    \item \url{https://www.freepik.com/free-photos-vectors/background}, Background vector created by freepik
\end{itemize}


\newpage
\begin{coverpage}
    \thispagestyle{empty}
    \noindent\fboxsep=0pt
    \ThisTileWallPaper{\paperwidth}{\paperheight}{Figures/backcover.png}
\end{coverpage}


\end{document}